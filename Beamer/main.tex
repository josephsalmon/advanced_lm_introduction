\documentclass[unknownkeysallowed]{beamer}
\usepackage[french,english]{babel}
\usepackage{beamer_js}
\usepackage{shortcuts_js}

% % package for JS compilation:
% \usepackage{/home/jo/Documents/OrganizationFiles/tex/sty/beamer_js}
% \usepackage{/home/jo/Documents/OrganizationFiles/tex/sty/shortcuts_js}
\usepackage{csquotes}

\graphicspath{{./prebuiltimages/}}

\addbibresource{Bibliographie.bib}
\usepackage{enumerate}

\begin{document}


%%%%%%%%%%%%%%%%%%%%%%%%%%%%%%%%%%%%%%%%%%%%%%%%%%%%%%%%%%%%%%%%%%%%%%%%%%%%%%%
%%%%%%%%%%%%%%%%%%%%%%             Headers               %%%%%%%%%%%%%%%%%%%%%%
%%%%%%%%%%%%%%%%%%%%%%%%%%%%%%%%%%%%%%%%%%%%%%%%%%%%%%%%%%%%%%%%%%%%%%%%%%%%%%%



%%%%%%%%%%%%%%%%%%%%%%%%%%%%%%%%%%%%%%%%%%%%%%%%%%%%%%%%%%%%%%%%%%%%%%%%%%%%%%%
\begin{frame}
\bigskip
\bigskip
\begin{center}{
\LARGE\color{marron}
\textbf{HMMA 307 : Advanced Linear Modeling}
\textbf{ }\\
\vspace{0.5cm}
}

\color{marron}
\textbf{Chapter 1 : Linear regression}
\end{center}

\vspace{0.5cm}

\begin{center}
\textbf{Emma Santinelli \ Mégane Diéval \ Yassine Sayd} \\
\vspace{0.1cm}
\url{https://github.com/MegDie/advanced_lm_introduction}\\
\vspace{0.5cm}
Université de Montpellier \\
\end{center}

\centering
\includegraphics[width=0.13\textwidth]{umontpellier_logo}

\end{frame}
%%%%%%%%%%%%%%%%%%%%%%%%%%%%%%%%%%%%%%%%%%%%%%%%%%%%%%%%%%%%%%%%%%%%%%%%%%%%%%%



%%%%%%%%%%%%%%%%%%%%%%%%%%%%%%%%%%%%%%%%%%%%%%%%%%%%%%%%%%%%%%%%%%%%%%%%%%%%%%%
%%%%%%%%%%%%%%%%%%%%%%%%       PLAN      %%%%%%%%%%%%%%%%%%%%%%%%%%%%%%%%%%%%%%
%%%%%%%%%%%%%%%%%%%%%%%%%%%%%%%%%%%%%%%%%%%%%%%%%%%%%%%%%%%%%%%%%%%%%%%%%%%%%%%



%%%%%%%%%%%%%%%%%%%%%%%%%%%%%%%%%%%%%%%%%%%%%%%%%%%%%%%%%%%%%%%%%%%%%%%%%%%%%%%
\begin{frame}{Table of Contents}
\tableofcontents[hideallsubsections]
\end{frame}
%%%%%%%%%%%%%%%%%%%%%%%%%%%%%%%%%%%%%%%%%%%%%%%%%%%%%%%%%%%%%%%%%%%%%%%%%%%%%%%



%%%%%%%%%%%%%%%%%%%%%%%%%%%%%%%%%%%%%%%%%%%%%%%%%%%%%%%%%%%%%%%%%%%%%%%%%%%%%%%
\AtBeginSection[]
{
\begin{frame}<beamer>{Table of Contents}
\tableofcontents[currentsubsection,
    hideothersubsections,
    sectionstyle=show/shaded,
]
\end{frame}
}
%%%%%%%%%%%%%%%%%%%%%%%%%%%%%%%%%%%%%%%%%%%%%%%%%%%%%%%%%%%%%%%%%%%%%%%%%%%%%%%




%%%%%%%%%%%%%%%%%%%%%%%%%%%%%%%%%%%%%%%%%%%%%%%%%%%%%%%%%%%%%%%%%%%%%%%%%%%%%%%
%%%%%%%%%%%%%%%%%%%%%%%%%%%%%%%%%%%%%%%%%%%%%%%%%%%%%%%%%%%%%%%%%%%%%%%%%%%%%%%
\section{Introduction and Ordinary Least Squares}
\label{sec:introdcution}
%%%%%%%%%%%%%%%%%%%%%%%%%%%%%%%%%%%%%%%%%%%%%%%%%%%%%%%%%%%%%%%%%%%%%%%%%%%%%%
%%%%%%%%%%%%%%%%%%%%%%%%%%%%%%%%%%%%%%%%%%%%%%%%%%%%%%%%%%%%%%%%%%%%%%%%%%%%%%%

%%%%%%%%%%%%%%%%%%%%%%%%%%%%%%%%%%%%%%%%%%%%%%%%%%%%%%%%%%%%%%%%%%%%%%%%%%%%%%%

%%%%%%%%%%%%%%%%%%%%%%%%%%%%%%%%%%%%%%%%%%%%%%%%%%%%%%%%%%%%%%%%%%%%%%%%%%%%%%%

%%%%%%%%%%%%%%%%%%%%%%%%%%%%%%%%%%%%%%%%%%%%%%%%%%%%%%%%%%%%%%%%%%%%%%%%%%%%%%%
\begin{frame}
\begin{alertblock}{Model}
Suppose the data consists of $n$ samples $( y_i, x_i )^n_{i=1}$ with $p$ features.
\newline The model can be written in matrix notation as :
\begin{center}
$y=X\beta+\epsilon$
\end{center}
where
 \begin{itemize}
        \item $X$ is an $n$$\times$$p$ matrix of regressors
        \item $\beta$  is a $p$$\times$1 vector of unknown parameters
        \item $\epsilon$ is a vector of normal random errors with mean 0
    \end{itemize}
\end{alertblock}
\vspace{0.4cm}
\end{frame}
\begin{frame}
The OLS estimator is any coefficient vector
$\hat\beta^{LS}\in\mathbb{R}^p$ such that :
\newline
\begin{center}
$\hat\beta^{LS} \in argmin \ \underbrace{\frac{1}{2n}\|y-X\beta\|^2 }_{f(\beta)}$
\end{center}
\vspace{0.5cm}
and $f(\beta)=\frac{1}{2n}\sum\limits_{i=1}^n (y_{i}-\frac{1}{2n}(X\beta)_{i})^2$ =\ $\beta^T\frac{X^TX}{2n}\beta+\frac{1}{2n}\|y\|^2- \langle y,X\beta\rangle$

\vspace{0.5cm}
where
$\langle y,X\beta\rangle=y^TX\beta=\beta^TX^Ty=\langle \beta,X^Ty\rangle$


\end{frame}
\begin{block}{Notation}
The matrix $\hat\Sigma=\frac{X^TX}{n}$ matrix is called the Gram matrix.
\begin{center}
    $X^TX=\begin{pmatrix}
   x_{1}^T  \\
   . \\
   . \\
   x_{p}^T  \\
\end{pmatrix}(x_{1}. . . x_{p}) $,
\end{center}

\end{block}
The Gram matrix is equivalent to :
\begin{center}
$[X^TX]_{j,j'}=[\langle x_{j},x_{j'}\rangle]_{(j,j')\in[1,p]^2}$
\end{center}

\begin{block}{Remark}
Most of the times, we scale features.
\newline
We have : $\bar{X_{j}}=\frac{1}{n} \sum\limits_{i=1}^{n} x_{ij}$ (1)
\newline
 To center explanatory variables, we use the equation (1) to build the centered vector $X_{c}$
\newline
$X_{c}$ =  $X$ - ($\bar{X_{1}}1_n,....,\bar{X_{p}}1_n$) where 1_n=(1,...,1)

Then we obtain $\bar{X_{c}}=O_n$

To reduce explanatory variables, we use :
\newline
\begin{center}
$\hat\sigma_{j}^2=\frac{1}{n} \sum\limits_{i=1}^{n} (X_{ij}-\bar{X_{j}})$
\end{center}
Let $X_r$ be the reduced vector, then :
\newline
\begin{center}
$X_{r_{j}}=\frac{X_{j}-\bar{X_{j}}1_n}{\hat\sigma_{j}}$
\end{center}

\end{block}


\begin{frame}
\begin{alertblock}{First Order Optimality Conditions}
We can verify the first order optimality condition because $\nabla{f(\hat\beta^{LS})}=0$
\\
Note that $f$ is a $C^{\infty}$ function, then differentiable

\end{alertblock}

\\\begin{block}{Remark}
$f$ is a convex function so a local minimum is a global one.
\end{block}

\begin{block}{Conclusion}
$\hat\beta^{LS}$ satisfy the following equations of orthogonality :
\begin{itemize}
        \item $\frac{X^TX}{n}\hat\beta^{LS}-\frac{X^Ty}{n}=0$
        \item \iff $X^T(\frac{X\hat\beta^{LS}-y}{n})=0$
        \item \iff $X^T(y-X\hat\beta^{LS})=0$
        \item \iff $\langle X_{j},y-X\beta\rangle=0$ for j in 1:p
    \end{itemize}


\end{block}

\newpage

\begin{block}{Attention}

If $p$ < $n$ so rank($X$) \leq $n$ < p Then $\hat\beta^{LS}$ is not unique

\end{block}


\end{frame}

\begin{frame}

\begin{alertblock}{Interpretation}
    \begin{itemize}
        \item  Each explanatory feature is orthogonal to the residuals $\Gamma=y-X\hat{\beta}^{LS}$ With $\hat{\beta}^{LS}$ a solution of the linear $p$$\times$$p$ system : $$ \hat{\Sigma}\beta= \frac{X^T y}{n}  $$

    \end{itemize}


\end{alertblock}

\begin{block}{Remarks}
    \begin{itemize}
        \item  If $\hat{\Sigma}$ is invertible, the solution of the linear system is unique
        \item $\hat{\Sigma}$ is invertible $\Rightarrow$ $\hat{\Sigma}$ is positive definite
        \item If $\hat{\Sigma}$ invertible, so $rank(\hat{\Sigma})=p$
        \item we assume that we have a full rank column e.g. : $$rank(X)=dim(Vect(X_1,...,X_p)) \leq n $$

    \end{itemize}


\end{block}
\end{frame}



\begin{frame}



\begin{alertblock}{Remark}
    \begin{itemize}
        \item  If $rank(X)=p$, so $\hat{\Sigma}$ is invertible and : $$\hat{\beta}^{LS}=\hat{\Sigma}^{-1}\frac{X^T y}{n}=(\frac{X^T y}{n})^{-1} \frac{X^T y}{n}$$
       so :  $$ \hat{\beta}^{LS}=(X^TX)^{-1}X^Ty$$

    \end{itemize}


\end{alertblock}

\begin{block}{Notice}
    \begin{itemize}
        \item  In practice it is exceptional to invert $\hat{\Sigma} $ because one solves many linear systems

    \end{itemize}


\end{block}
\end{frame}

\begin{frame}{Data analysis}

\begin{block}{Goal}
    We want to build some ordinary least squares models of prediction with two datasets:

    \begin{itemize}
          \item Bicycle accidents
          \item Count data of bicycles
      \end{itemize}
\end{block}


\vspace{0.5cm}

We propose to estimate the severity of accidents by the feature "sexe". The problem is that the features are qualitative:
\begin{itemize}
    \item Modalities of the feature to predict: "0 - Indemne", "1 - Blessé léger", "Blessé hospitalisé", and "3 - Tué"
    \item Modalities of the feature "sexe": "M" and "F"
\end{itemize}


\end{frame}

\begin{frame}{Data analysis}

\begin{alertblock}{Solution}
We convert features into ordinal features.
\end{alertblock}

\begin{block}{Prediction principle}

Calculate the coefficients $\beta$ on a training sample and predict on a test sample the feature of interest. 0 is the value for male and 1 is the value for female.


\end{block}

\begin{minipage}[c]{.36\linewidth}
     \begin{center}
             \includegraphics[width=5.5cm]{stat_model_gravity}
         \end{center}
   \end{minipage} \hfill
   \begin{minipage}[c]{.55\linewidth}
    \begin{center}
            \includegraphics[width=5.5cm]{severitypredictionwithsex}

        \end{center}

 \end{minipage}

\end{frame}

\begin{frame}{Data analysis}

\begin{block}{Conclusion}
The prediction is very bad on qualitative features. We notice that the $R^2$ is closed to 0 and it's mostly the same with the others qualitative features. With this dataset, the OLS model is not efficient for qualitative features.

\end{block}

\begin{block}{Prediction of a quantitative feature}
Predict the number of accidents with the date (day, month and year) that is an ordinal feature with periodic component. Results are also very bad.

\end{block}

\begin{minipage}[c]{.36\linewidth}
     \begin{center}
             \includegraphics[width=4cm]{stat_model_number1}
         \end{center}
   \end{minipage} \hfill
   \begin{minipage}[c]{.55\linewidth}
    \begin{center}
            \includegraphics[width=5.5cm]{accidentprediction}

        \end{center}

 \end{minipage}

\end{frame}

\begin{frame}{Data analysis}

\begin{block}{Same thing on the second dataset}
Prediction of the number of bicycles in a day with the date and the total number of bicycles. We introduce also periodic components.
\end{block}


\begin{minipage}[c]{.36\linewidth}
     \begin{center}
             \includegraphics[width=4cm]{stat_model_albert}
         \end{center}
   \end{minipage} \hfill
   \begin{minipage}[c]{.55\linewidth}
    \begin{center}
            \includegraphics[width=5.5cm]{accidentpredictionalbert1}

        \end{center}

 \end{minipage}




\end{frame}





%%%%%%%%%%%%%%%%%%%%%%%%%%%%%%%%%%%%%%%%%%%%%%%%%%%%%%%%%%%%%%%%%%%%%%%%%%%%%%%



%%%%%%%%%%%%%%%%%%%%%%%%%%%%%%%%%%%%%%%%%%%%%%%%%%%
%%%%%%%%%%%%%%%%%%%%%%%%%%%%%%%%%%%%%%%%%%%%%%%%%%%%%%%%%%%%%%%%%%%%%%%%%%%%%%%



%%%%%%%%%%%%%%%%%%%%%%%%%%%%%%%%%%%%%%%%%%%%%%%%%%%%%%%%%%%%%%%%%%%%%%%%%%%%%%%

%%%%%%%%%%%%%%%%%%%%%%%%%%%%%%%%%%%%%%%%%%%%%%%%%%%%%%%%%%%%%%%%%%%%%%%%%%%%%%%
%%%%%%%%%%%%%%%%%%%%%%%%%%%%%%%%%%%%%%%%%%%%%%%%%%%%%%%%%%%%%%%%%%%%%%%%%%%%%%%
\section{Singular Value Decomposition}
\label{sec:conclusion}
%%%%%%%%%%%%%%%%%%%%%%%%%%%%%%%%%%%%%%%%%%%%%%%%%%%%%%%%%%%%%%%%%%%%%%%%%%%%%%%
%%%%%%%%%%%%%%%%%%%%%%%%%%%%%%%%%%%%%%%%%%%%%%%%%%%%%%%%%%%%%%%%%%%%%%%%%%%%%%%

\begin{frame}
\begin{block}{Reminder}
Let $\Sigma\in\mathbb{R}^{p \times p}$.
\newline
If   $\Sigma^T=\Sigma$ then $\Sigma$ is diagonalizable.
\end{block}
\begin{alertblock}{Theorem}
For all matrix $M\in\mathbb{R}^{m_1 \times m_2}$ of rank $r$, there exist two orthogonal matrix $U\in \mathbb{R}^{m_1 \times r}$ and $V\in\mathbb{R}^{m_2 \times r}$ such that :
\begin{center}
    $M=U diag(s_{1}...s_{r})U^T$
\end{center}
where $s_{1}\ge s_{2} \ge ... \ge s_{r} \ge 0$ are the singular values of M.
\end{alertblock}
\vspace{0.5cm}
Note that : $M=\sum\limits_{j=1}^r s_{j}u_{j}v_{j}^T$ with : $U=[u_{1},...,u_{r}]$
et $V=[v_{1} ... v_{r}]$
\end{frame}
%%%%%%%%%%%%%%%%%%%%%%%%%%%%%%%%%%%%%%%%%%%%%%%%%%%%%%%%%%%%%%%%%%%%%%%%%%%%%%%

\begin{frame}
\begin{block}{Definition}
For $M\in\mathbb{R}^{m_1 \times m_2}$, a pseudoinverse of $M$ is defined as a matrix $M^{+}$ satisfying :
\begin{center}
    $M^{+}=Vdiag(\frac{1}{s_{1}} ... \frac{1}{s_{r}})U^T$
    =$\sum\limits_{j=1}^r \frac{1}{s_{j}}v_{j}u_{j}^T$
\end{center}
\end{block}
\vspace{0.5cm}
Remark : If $M$ is invertible, its pseudoinverse is its inverse. That is, $A^{+}=A^{-1}$
\end{frame}

%%%%%%%%%%%%%%%%%%%%%%%%%%%%%%%%%%%%%%%%%%%%%%%%%%%%%%%%%%%%%%%%%%%%%%%%%%%%%%%
\begin{frame}{Bibliography}
[1] Joseph Salmon, \textit{Modéle linéaire avancé : introduction}, 2019, \url{http://josephsalmon.eu/enseignement/Montpellier/HMMA307/Introduction.pdf}.
\newline

[2] Francois Portier and Anne Sabourin, \textit{Lecture notes on ordinary least squares}, 2019, \url{https://perso.telecom-paristech.fr/sabourin/mdi720/main.pdf}
\newline
\newline
[3] \textit{Ordinary least squares}, 2020,
\url{https://en.wikipedia.org/wiki/Ordinary_least_squares}.
\newline
\newline
[4] \textit{Singular value decomposition}, 2020,
\url{https://en.wikipedia.org/wiki/Singular_value_decomposition}.


\newline
\printbibliography
\end{frame}
 %%%%%%%%%%%%%%%%%%%%%%%%%%%%%%%%%%%%%%%%%%%%%%%%%%%%%%%%%%%%%%%%%%%%%%%%%%%%%%



\end{document}